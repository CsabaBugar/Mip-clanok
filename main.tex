\documentclass[12pt,english]{article}
\usepackage[utf8]{inputenc}
\usepackage{hyperref}
\pagenumbering{gobble}


\title{The different approaches towards online education }
\author{Csaba Bugár}
\date{08 October 2020}



\begin{document}

\maketitle

Taking today's situation into account, online education is about to replace the traditional forms of acquiring knowledge. However online education is not entirely a new invention, it has been around for decades in various different ways. The question is, whether we can benefit from the technology? To answer that question, first of all we have to understand that as of today you can enrich your knowledge using a great variety of ways and methods to approach the learning process itself. Therefore to compare it effectively we need to group the different methods. Thus  we can categorize them based on the independence of the tutee, whether the content is fixed or adaptive, the medium of the materials, and lastly based on the role of the computer or application.


\section*{}
Moja inšpirácia: \url{https://www.fordham.edu/info/24884/online_learning/7897/types_of_online_learning}



\end{document}
